%++++++++++++++++++++++++++++++++++++++++
% Don't modify this section unless you know what you're doing!
\documentclass[10pt, a4paper]{article}


\usepackage[latin1]{inputenc} %g�re les caract�res accentu�s et symboles sp�cifiques de l?alphabet d?une langue
\usepackage[T1]{fontenc}%gere les polices de caract�res (definition de la position des caract�res dans une police)
\usepackage{lmodern}%police vectorielle
\usepackage{hyperref}%transforme toutes les r�f�rences internes du document en hyperliens
\usepackage{fancyhdr}%pour des ent�tes am�lior�s
\usepackage{lastpage}%pour pouvoir utiliser 
\usepackage{xcolor,graphicx}%pour des documents en couleur et ins�rer des images
\usepackage{pgf,tikz}%package permettant d?inclure des figures au format PDF en restant dans l?environnement LATEX.
\usepackage{pgfplots}%pour tracer des courbes/graphiques
\usepackage{framed}%minipages avec cadre autour
\bibliographystyle{ieeetr}%style de bibliographie
\usetikzlibrary{arrows}%dessin de fleches
\usepackage{caption}%permet de customiser les l�gendes dans des environnements flottants
\usepackage{amsmath,amssymb,color,bm}%charge le formulaire mathematique+symb+caracteres gras  avec \boldsymbol en restant un caract mathematique en italique
\usepackage{ulem}
\usepackage{layout}%permet de tracer la mise en page avec \layout
\usepackage{cite} % takes care of citations
\newcommand*{\Coord}[3]{% 
  \ensuremath{\overrightarrow{#1}\, 
    \begin{pmatrix} 
      #2\\ 
      #3 
    \end{pmatrix}}}

%miseenpage
\setlength{\oddsidemargin}{0cm}
\setlength{\topmargin}{0cm}
\setlength{\textheight}{24.7cm}
\setlength{\marginparwidth}{0cm}
\setlength{\textwidth}{15.8cm}
\setlength{\marginparsep}{0pt}
\setlength{\voffset}{-1cm}
\setlength{\hoffset}{-0.29cm}
\setlength{\headheight}{14.5pt} %pas vraiment n�c�ssaire mais cr�e des warnings sans...

%\setlength{\parindent}{0pt} %indentation


\setlength{\oddsidemargin}{0cm}
\setlength{\topmargin}{-1cm}
\setlength{\textheight}{24cm}
\setlength{\marginparwidth}{0.2cm}
\setlength{\textwidth}{16cm}
\setlength{\voffset}{-0.5cm}
\setlength{\headheight}{14.5pt} %pas vraiment n�c�ssaire mais cr�e des warnings sans...
%++++++++++++++++++++++++++++++++++++++++


\begin{document}

\title{Assignment 1 - Network Graph Sciences Analysis}
\author{Florian \textsc{Barral}}

\date{\today}
\maketitle

%\begin{abstract}
%\end{abstract}


\section*{Question 1}
\textbf{(a)} 
$ \textbf{k}=A . 1$ \\
\textbf{(b)} 
$m=|E|= \frac12 k^Tk = \frac12 \sum_{i=0}^n k_i$\\
\textbf{(c)}
$N=AA^T$,
\newpage

\section*{Question 2}
Let $V$ be partitioned into two disjoint sets $V_1$, $V_2$.
Because there are no internal edges in each set $V_1$, $V_2$ :
\begin{align*}
\{ \# \text{ of out edges of } V_1\} &=\{ \# \text{ of out edges of } V_2\} \\
\sum_{i\in V_1} k_i & = \sum_{i\in V_2}  k_i \\
{|V_1| }\frac{\sum_{i\in V_1} k_i}{|V_1|} & = {|V_2| }\frac{\sum_{i\in V_2} k_i}{|V_2|}i \\
{|V_1|}c_1 & ={|V_2|}c_2 \\
n_1 c_1 & =n_1 c_1 \\
\end{align*}
\newpage
\section*{Question 3}
Graph : G=(V,E)\\
\textbf{(a)} 
By definition, the element $A_{ij}$ of the square matrix contains the number of paths of length 3 that starts from node $i$ and end at node $j$. A triangle is designed when the node participates simultaneously as the start node and the end node, so we are interested in diagonal elements $A_{ii}$ : nodes involved in triangles.\\ Therefore, the trace of $A^3$ is related to the number of triangles. As :\\
- 3 nodes are involved in a triangle : we are then triple counting the number of triangles\\
- the graph is undirected : we are then doubling the number of triangles\\
So to conclude, the total number of triangles in the graph is 
$$\Delta(G)=\frac16 tr(A^3)$$
\\
\textbf{(b)} 
As we know the trace of a matrix if the sum of eigen values. We have :
$$\Delta(G)=\frac16 \sum_{i \in V} \lambda_i^3$$

\textbf{(c)} 
As we have just shown that the diagonal element $A_{ii}$ is $\Delta_i$ the number of triangles that node $i$ participates in, we just need to find a way to derive $A_{ii}$. The adjacency matrix $A$ being real and symmetric, we have $A=U\Sigma U^T$ with $UU^T=I$. Then $A^3=U\Sigma^3 U^T$. 
So with the eigenvectors $(\vec{u_{1}},...~,\vec{u_{n}})$ of the eigenvalues $(\lambda_{1},...,\lambda_{n})$, and $u_{ij}$ the $i$th element of $\vec{u_{j}}$, we can derive $\Delta_i$ as follows:

\begin{align*}
\Delta_i=\frac12 A^3_{ii} &=\frac12 \sum_{k\in V} u_{ik} (\Sigma^3U)_{ki}\\
& =\frac12 \sum_{k\in V} u_{ik} (\sum_{l\in V} \Sigma^3_{kl}u_{li})\\
& =\frac12 \sum_{k\in V} u_{ik} (\lambda_k^3 u_{ki})\\
\Delta_i & =\frac12 \sum_{k\in V} \lambda_k^3 u_{ki}^2 \\
\end{align*}
\newpage

\section*{Question 4}
\textbf{(a)} 
$G_{n,p}$undirected graph on $n$ nodes and each edge $(u, v)$ appears i.i.d. with probability $p$
\textbf{(a)} 
There are $\binom{n}{3}$ triples of vertices . Each triple has statistically a probability of $p$ of being a triangle. Let $\Delta_{ijk}$ be the random variable for the triangle with vertices i, j, and k being present equals to 1 if vertices participate to the same triangle. Then the number of triangles is the expectancy of $\sum_{ijk}\Delta_{ijk}$:
\begin{align*}
E[\sum_{ijk}\Delta_{ijk}]=* \sum_{ijk}E[\Delta_{ijk}]=\sum_{ijk}p^3 &= \binom{n}{3}p^3\\
&=  \frac{n(n-1)(n-2)}{3!}p^3\\
& \underset{\infty}{\sim} \frac16(np)^3=\frac16 c^3
\end{align*}
*the expected value of a sum of random variables is the sum of the expected values, because the events are identically distributed.\\
\textbf{(b)} 
Mutatis mutandis :
\begin{align*}
E[\sum_{ijk}\Delta'_{ijk}]=* \sum_{ijk}E[\Delta'_{ijk}]=\sum_{ijk}  \binom{3}{2} p^2 &= \binom{n}{3} \binom{3}{2}p^2\\
&=  \frac{n(n-1)(n-2)}{3!} \frac{3!}{2!}p^2\\
& \underset{\infty}{\sim} \frac12nc^2
\end{align*}
\textbf{(c)} 
$C=\frac{ \frac16 c^3 3 }{\frac12nc^2}=\frac{c}{n} \sim p$
\newpage


\section*{Question 5}
\textbf{(a)} 
$$
x_i=\sum_{k=1}^{\infty}\sum_{j \in V} \alpha^k A^k_{ij}
$$
\textbf{(b)} 
As we know above : 
\begin{align*}
A^k_{ij} = \sum_{l\in V} \lambda_l^k u_{li} u_{lj} \\
\end{align*}
\newpage

\section*{Question 6}
Let's reason by equivalence to prove $ \boxed{\frac{1}{C_A}+\frac{n_A}{n} = \frac{1}{C_B}+\frac{n_B}{n}}$ : 
\begin{align*}
\frac{1}{C_A}+\frac{n_A}{n} &= \frac{1}{C_B}+\frac{n_B}{n}\\
\iff \frac{\sum_{j}d_{Aj}}{n}+\frac{n_A}{n}& =\frac{\sum_{j}d_{Bj}}{n}+\frac{n_B}{n} \\
\iff {\sum_{j}d_{Aj}}+{n_A}& ={\sum_{j}d_{Bj}}+{n_B} \\
\iff \sum_{j \in A \cup B}d_{Aj}+n_A & = \sum_{j \in A \cup B}d_{Bj}+n_B \\
\iff \sum_{j \in A}d_{Aj}+\sum_{j \in B}d_{Aj}+n_A & = \sum_{j \in A}d_{Bj}+ \sum_{j \in B}d_{Bj}+n_B \\
\iff \sum_{j \in A}d_{Aj}+\sum_{j \in B}(d_{Bj}+1)+n_A & = \sum_{j \in A}(d_{Aj}+1)+ \sum_{j \in B}d_{Bj}+n_B \\
\iff \sum_{j \in A}d_{Aj}+\sum_{j \in B}d_{Bj}+n_B+n_A & = \sum_{j \in A}d_{Aj}+n_A+ \sum_{j \in B}d_{Bj}+n_B  ~~~ \text{which is true} \\
\end{align*}
\newpage

\section*{Question 9}
\textbf{(a)} 
The produced Kronecker graph is connected, if: 
$$ b+c > 1$$
\\ Here $b+c=0.26+0.53=0.79$ . So the produced Kronecker graph is NOT connected. \\ \\
The graph has a giant connected component of size $\Theta(n)$ if:
$$(a+b)(b+c) > 1$$
Here $(a+b)(b+c)=(0.99+0.26)(0.26+0.53)=0.9875$ . So the produced Kronecker graph is NOT connected. \\

\textbf{(b)} 
Similar structural properties : 
\begin{itemize}
\item Triangle participation distribution
\item Visualization of nodes and edges of graph.
\end{itemize}

\end{document}

%\mathbf M=(M_{i,j})_{(i,j) \in{[\![ 1,n]\!]\times[\![ 1,m]\!]}},~~~  \mathbf {MM^T}=((MM^T)_{i,j})_{(i,j) \in{[\![ 1,n]\!]^2}},~~~  \mathbf {M^TM}=((M^TM)_{i,j})_{(i,j) \in{[\![ 1,n]\!]^2}}
% \begin{align*}
%\forall (i,j) \in [\![ 1,n]\!]^2, ~~~~~
%((MM^T)_{i,j} & = \sum_{k=1}^{m} M_{i,k} M^T_{k,j} \\
%& = \sum_{k=1}^{m} M_{i,k} M_{j,k} 
%\end{align*}

%\begin{align*}
%\forall (i,j) \in [\![ 1,n]\!]^2, ~~~~~
%(M^TM)_{i,j} & = \sum_{k=1}^{m} M^T_{i,k} M_{k,j} \\
%& = \sum_{k=1}^{m} M_{k,i} M_{k,j} 
%\end{align*}